\documentclass{article}

\usepackage{tabularx}
\usepackage{booktabs}

\title{Problem Statement and Goals\\Binary Star System Simulator}

\author{Xinlu Yan}

\date{Jan 15, 2026}

\input{../Comments.text}
\input{../Common.text}

\begin{document}

\maketitle

\begin{table}[hp]
\caption{Revision History} \label{TblRevisionHistory}
\begin{tabularx}{\textwidth}{llX}
\toprule
\textbf{Date} & \textbf{Developer(s)} & \textbf{Change}\\
\midrule
Jan 15, 2026 & Xinlu Yan & Initial document\\
\bottomrule
\end{tabularx}
\end{table}

\section{Problem Statement}

\subsection{Problem}

Binary stars are very common in space. Since two stars orbit each other due to gravity,
studying them helps us understand how mass is distributed and how orbits stay stable.
For this project, the goal is to create a model that shows how these stars move over time.

The main challenge is that gravity is constantly changing as the stars move, so we can't
just use a simple formula to find their positions at any time. Instead, we need to use
a computer program to calculate the orbit step-by-step. The objective of this project
is to develop a reliable software tool that can take initial data (like mass and velocity)
and accurately simulate how the binary system evolves.


\subsection{Inputs and Outputs}

The system's inputs consist of the initial setup of the binary star system. This includes
the essential physical properties of both celestial bodies and their starting state in space.
Additionally, the inputs specify the time parameters that determine the duration and resolution
of the simulation.

The output is a time-ordered history of the system’s configuration. It provides a sequence of
the orbital states for both bodies, showing how their positions and movements change over the
requested time frame. This data serves as the basis for analyzing the system's overall stability
and behavior.

\subsection{Stakeholders}

The primary stakeholders of this project include:

\begin{itemize}
  \item \textbf{Astronomy Students and Educators}: Those using the simulation as a learning tool to
  visualize orbital mechanics and understand how gravitational systems work.
  \item \textbf{Astrophysics Researchers}: People who need a reliable tool to test different orbital
  scenarios and observe the qualitative behavior of binary systems.
  \item \textbf{Scientific Software Developers}: Future developers or students who will take over the 
  code to add new features or integrate it into larger educational platforms.
\end{itemize}


\subsection{Environment}

The software is designed to run on standard personal computers and does not require specialized hardware.
It is intended for use in typical educational and research settings, such as classrooms or labs.
The system relies on a standard scientific computing environment and produces outputs in formats compatible
with common data analysis and visualization tools, so users can readily plot and inspect the results.

\section{Goals}

\begin{itemize}
  \item To provide a clear and well-defined formulation of the problem of modeling the orbital evolution
  of a binary star system.
  \item To generate predictions of system behavior that are consistent with established physical principles.
  \item To support the systematic generation of software artifacts, including requirements documentation and
  a reference implementation, in a consistent and maintainable manner.
\end{itemize}

\section{Stretch Goals}

\begin{itemize}
  \item Provide visual representations of orbital trajectories, such as plots or animations, to enhance
  interpretability for educational and research use.
  \item Support a wider range of initial configurations, including extreme scenarios, to explore the
  robustness and limitations of the model.
\end{itemize}


\end{document}