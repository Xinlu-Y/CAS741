% THIS DOCUMENT IS TAILORED TO REQUIREMENTS FOR SCIENTIFIC COMPUTING.  IT SHOULDN'T
% BE USED FOR NON-SCIENTIFIC COMPUTING PROJECTS
\documentclass[12pt]{article}

\usepackage{amsmath, mathtools}
\usepackage{amsfonts}
\usepackage{amssymb}
\usepackage{graphicx}
\usepackage{colortbl}
\usepackage{xr}
\usepackage{hyperref}
\usepackage{longtable}
\usepackage{xfrac}
\usepackage{tabularx}
\usepackage{float}
\usepackage{siunitx}
\usepackage{booktabs}
\usepackage{caption}
\usepackage{pdflscape}
\usepackage{afterpage}

\usepackage[numbers]{natbib}

%\usepackage{refcheck}

\hypersetup{
    bookmarks=true,         % show bookmarks bar?
      colorlinks=true,       % false: boxed links; true: colored links
    linkcolor=red,          % color of internal links (change box color with linkbordercolor)
    citecolor=green,        % color of links to bibliography
    filecolor=magenta,      % color of file links
    urlcolor=cyan           % color of external links
}

\input{../Comments.text}
\input{../Common.text}

% For easy change of table widths
\newcommand{\colZwidth}{1.0\textwidth}
\newcommand{\colAwidth}{0.13\textwidth}
\newcommand{\colBwidth}{0.82\textwidth}
\newcommand{\colCwidth}{0.1\textwidth}
\newcommand{\colDwidth}{0.05\textwidth}
\newcommand{\colEwidth}{0.8\textwidth}
\newcommand{\colFwidth}{0.17\textwidth}
\newcommand{\colGwidth}{0.5\textwidth}
\newcommand{\colHwidth}{0.28\textwidth}

% Used so that cross-references have a meaningful prefix
\newcounter{defnum} %Definition Number
\newcommand{\dthedefnum}{GD\thedefnum}
\newcommand{\dref}[1]{GD\ref{#1}}
\newcounter{datadefnum} %Datadefinition Number
\newcommand{\ddthedatadefnum}{DD\thedatadefnum}
\newcommand{\ddref}[1]{DD\ref{#1}}
\newcounter{theorynum} %Theory Number
\newcommand{\tthetheorynum}{TM\thetheorynum}
\newcommand{\tref}[1]{TM\ref{#1}}
\newcounter{tablenum} %Table Number
\newcommand{\tbthetablenum}{TB\thetablenum}
\newcommand{\tbref}[1]{TB\ref{#1}}
\newcounter{assumpnum} %Assumption Number
\newcommand{\atheassumpnum}{A\theassumpnum}
\newcommand{\aref}[1]{A\ref{#1}}
\newcounter{goalnum} %Goal Number
\newcommand{\gthegoalnum}{GS\thegoalnum}
\newcommand{\gsref}[1]{GS\ref{#1}}
\newcounter{instnum} %Instance Number
\newcommand{\itheinstnum}{IM\theinstnum}
\newcommand{\iref}[1]{IM\ref{#1}}
\newcounter{reqnum} %Requirement Number
\newcommand{\rthereqnum}{R\thereqnum}
\newcommand{\rref}[1]{R\ref{#1}}
\newcounter{nfrnum} %NFR Number
\newcommand{\rthenfrnum}{NFR\thenfrnum}
\newcommand{\nfrref}[1]{NFR\ref{#1}}
\newcounter{lcnum} %Likely change number
\newcommand{\lthelcnum}{LC\thelcnum}
\newcommand{\lcref}[1]{LC\ref{#1}}

\usepackage{fullpage}

\newcommand{\deftheory}[9][Not Applicable]
{
\newpage
\noindent \rule{\textwidth}{0.5mm}

\refstepcounter{theorynum}
\paragraph{RefName: } \textbf{\tthetheorynum} \phantomsection 
\label{#2}

\paragraph{Label:} #3

\noindent \rule{\textwidth}{0.5mm}

\paragraph{Equation:}

#4

\paragraph{Description:}

#5

\paragraph{Notes:}

#6

\paragraph{Source:}

#7

\paragraph{Ref.\ By:}

#8

\paragraph{Preconditions for \hyperref[#2]{#2}:}
\label{#2_precond}

#9

\paragraph{Derivation for \hyperref[#2]{#2}:}
\label{#2_deriv}

#1

\noindent \rule{\textwidth}{0.5mm}

}

\begin{document}

\title{Software Requirements Specification for BSS:\\ A Binary Star System Simulator} 
\author{Xinlu Yan}
\date{\today}
	
\maketitle

~\newpage

\pagenumbering{roman}

\tableofcontents

~\newpage

\section*{Revision History}

\begin{tabularx}{\textwidth}{p{3cm}p{2cm}X}
\toprule {\bf Date} & {\bf Version} & {\bf Notes}\\
\midrule
Jan. 27, 2026 & 1.0 & Template imported.\\
Date 2 & 1.1 & Notes\\
\bottomrule
\end{tabularx}

~\\
\plt{This template is intended for use by CAS 741.  For CAS 741 the template
should be used exactly as given, except the Reflection Appendix can be deleted.
For the capstone course it is a source of ideas, but shouldn't be followed
exactly.  The exception is the reflection appendix.  All capstone SRS documents
should have a reflection appendix.}

~\newpage

\section{Reference Material}

This section records information for easy reference.

\subsection{Table of Units}

Throughout this document SI (Syst\`{e}me International d'Unit\'{e}s) is employed
as the unit system.  In addition to the basic units, several derived units are
used as described below.  For each unit, the symbol is given followed by a
description of the unit and the SI name.
~\newline

\renewcommand{\arraystretch}{1.2}
%\begin{table}[ht]
  \noindent \begin{tabular}{l l l} 
  \toprule		
  \textbf{symbol} & \textbf{quantity} & \textbf{SI name}\\
  \midrule 
  \si{\metre} & length & metre\\
  \si{\kilogram} & mass & kilogram\\
  \si{\second} & time & second\\
  \si{\metre\per\second} & velocity & metre per second\\
  \si{\metre\per\second\squared} & acceleration & metre per second squared\\
  \si{\newton} & force & newton\\
  \si{\metre\cubed\per\kilogram\per\second\squared} & derived (for $G$) & metre cubed per kilogram per second squared\\
  \bottomrule
  \end{tabular}
  %	\caption{Provide a caption}
%\end{table}

\subsection{Table of Symbols}

The table that follows summarizes the symbols used in this document along with
their units.
The choice of symbols was made to be consistent with the literature related to materials.

\renewcommand{\arraystretch}{1.2}
\noindent \begin{longtable*}{l l p{12cm}} \toprule
\textbf{symbol} & \textbf{unit} & \textbf{description}\\
\midrule
$G$ & \si{\metre\cubed\per\kilogram\per\second\squared} & gravitational constant \\
$m_1$ & \si{\kilogram} & mass of star 1 \\
$m_2$ & \si{\kilogram} & mass of star 2 \\
$\mathbf{r}_1(t)$ & \si{\metre} & position vector of star 1 at time $t$ (in $\mathbb{R}^2$) \\
$\mathbf{r}_2(t)$ & \si{\metre} & position vector of star 2 at time $t$ (in $\mathbb{R}^2$) \\
$\mathbf{r}_{12}(t)$ & \si{\metre} & relative position vector $\mathbf{r}_1(t)-\mathbf{r}_2(t)$ \\
$r_{12}(t)$ & \si{\metre} & separation distance $\lVert \mathbf{r}_{12}(t) \rVert$ \\
$\mathbf{v}_1(t)$ & \si{\metre\per\second} & velocity vector of star 1 at time $t$ \\
$\mathbf{v}_2(t)$ & \si{\metre\per\second} & velocity vector of star 2 at time $t$ \\
$\mathbf{a}_1(t)$ & \si{\metre\per\second\squared} & acceleration vector of star 1 at time $t$ \\
$\mathbf{a}_2(t)$ & \si{\metre\per\second\squared} & acceleration vector of star 2 at time $t$ \\
$\mathbf{F}_{12}$ & \si{\newton} & gravitational force on star 1 due to star 2 \\
$t$ & \si{\second} & time \\
$t_{\text{final}}$ & \si{\second} & end time of the simulation interval \\
\bottomrule
\end{longtable*}

\subsection{Abbreviations and Acronyms}

\renewcommand{\arraystretch}{1.2}
\begin{tabular}{l l} 
  \toprule		
  \textbf{symbol} & \textbf{description}\\
  \midrule 
  A & Assumption\\
  BSS & Binary Star System Simulator\\
  DD & Data Definition\\
  GD & General Definition\\
  GS & Goal Statement\\
  IM & Instance Model\\
  IVP & Initial Value Problem\\
  LC & Likely Change\\
  PS & Physical System Description\\
  R & Requirement\\
  SRS & Software Requirements Specification\\
  TM & Theoretical Model\\
  \bottomrule
\end{tabular}\\

\subsection{Mathematical Notation}
Vectors are denoted by bold symbols (e.g., $\mathbf{r}$ and $\mathbf{v}$).
The Euclidean norm of a vector $\mathbf{x}$ is written as $\lVert \mathbf{x} \rVert$.

\newpage
\pagenumbering{arabic}
\section{Introduction}

Binary star systems are common in astronomy. Two stars orbit because of gravity.
Studying their motion helps us understand basic mechanics and gravitational
interaction. Many binary systems can be simplified as two bodies affected only
by each other’s gravity.

Software can be used to simulate how a binary star system evolves over time.
Such simulations are useful for learning and for analysis. In this document, a
software artifact that meets these requirements is called Binary Star System Simulator (BSS).

This document presents the Software Requirements Specification (SRS) for
BSS. It explains the purpose of the document and the scope of the
requirements.

\subsection{Purpose of Document}

The purpose of this document is to describe what the software must do to
simulate a binary star system. It states the assumptions, inputs, and outputs.
It also helps developers build and verify the software.

This SRS supports traceability from the physical problem to the models and the
final software. It does not prescribe implementation details such as numerical
methods, programming languages, or tools.

\subsection{Scope of Requirements} \label{sec_Scope}

The requirements analyze the problem described in Section~\ref{Sec_pd} and the
related solution described in Section~\ref{sec_instance} under Newtonian gravitation. The
system is modeled as two point masses in a two-dimensional space, and the
software computes their motion from given initial conditions over a specified
time span.

The scope excludes effects such as relativity, external bodies, non-gravitational
forces, and changes in mass. These exclusions keep the problem simple and
focused on the two-body model.

\subsection{Characteristics of Intended Reader} \label{sec_IntendedReader}

This document targets readers with basic undergraduate physics and mathematics.
Readers are expected to understand Newton's laws, Newtonian gravity, and common
calculus and linear algebra concepts such as vectors, derivatives, and
integrals. Familiarity with introductory differential equations will help when
reading the parts that describe how the system changes over time.

A stronger background in classical mechanics, especially two-body motion and
conservation laws, will make it easier to review and audit the models and
assumptions.

This document is intended for educational use only and is provided ``as is,''
without warranty. The author accepts no liability for any damages arising from
its use.


\subsection{Organization of Document}
This document follows the SRS template as specified by Smith et al.~\citep{
SmithAndKoothoor2016, SmithAndLai2005, SmithEtAl2007}. 
If you are already familiar with the SRS template, the author’s recommended
reading order is:

\begin{itemize}
  \item \nameref{Sec_pd} (Section~\ref{Sec_pd})
  \item \nameref{sec_GoalStatements} (Section~\ref{sec_GoalStatements})
  \item \nameref{sec_IntendedReader} (Section~\ref{sec_IntendedReader})
  \item \nameref{sec_Scope} (Section~\ref{sec_Scope})
  \item \nameref{sec_GeneralSystemDescription} (Section~\ref{sec_GeneralSystemDescription})
  \item \nameref{sec_assumpt} (Section~\ref{sec_assumpt})
  \item \nameref{sec_SpecificSystemDescription} (Section~\ref{sec_SpecificSystemDescription})
  \item \nameref{sec_instance} (Section~\ref{sec_instance}) to \nameref{sec_theoretical} (Section~\ref{sec_theoretical})
\end{itemize}

The other material is referential and may be read as needed.
\newpage

\section{General System Description} \label{sec_GeneralSystemDescription}

This section provides general information about the system.  It identifies the
interfaces between the system and its environment, describes the user
characteristics and lists the system constraints.

\subsection{System Context}

Figure~\ref{Fig_SystemContext} shows an abstract view of the Binary Star System
simulation software. The rectangular node denotes the software, and the
circular nodes denote external entities.

The user provides the initial conditions and simulation settings. The software
computes the evolution of the binary system and returns the resulting star
trajectories. An external numerical integrator may be used to solve the
governing equations and returns either a solution or an error.

\begin{figure}[h!]
\begin{center}
 \includegraphics[width=0.6\textwidth]{SystemContextFigure}
\caption{System Context}
\label{Fig_SystemContext} 
\end{center}
\end{figure}

\begin{itemize}
  \item User Responsibilities:
    \begin{itemize}
      \item Providing the physical parameters of the binary star system, including stellar masses, initial positions, and initial velocities, taking required units, assumptions, and applicability of the physical model into consideration.
      \item Providing simulation settings such as time span.
      \item Interpreting the output of the program in accordance with this SRS document and using an audited and reliable copy of software related to this SRS.
    \end{itemize} 
\end{itemize}
\begin{itemize}
  \item BSS Responsibilities:
    \begin{itemize}
      \item Detecting data type mismatches, such as invalid numerical inputs or inconsistent parameter dimensions.
      \item Validating the applicability of the input arguments to the physical assumptions and models specified in this document.
      \item Computing the time evolution of the binary star system under gravitational interaction and producing the corresponding trajectories.
    \end{itemize} 
\end{itemize}
\begin{itemize}
  \item External IVP Solver Responsibilities:
    \begin{itemize}
      \item Attempting to solve systems of ordinary differential equations formulated as initial value problems on demand.
    \end{itemize}
\end{itemize}

\subsection{User Characteristics} \label{SecUserCharacteristics}

The user of this software is expected to have a basic understanding of classical
mechanics, including Newtonian gravitation, motion under forces, and related
concepts. If the software is used for educational purposes, an understanding
equivalent to first-year university physics and calculus is sufficient to
interpret the results and simulate a binary star system under simplified
conditions.

The author of this document accepts no liability and provides no warranty for any
usage of the software. The software should be appropriately audited by qualified
authorities and conform to applicable local laws and regulations.

\subsection{System Constraints}

The software shall be developed using Drasil to encode the problem and generate
a binary star system simulation based on an IVP formulation. No additional system constraints are imposed.
\newpage
\section{Specific System Description} \label{sec_SpecificSystemDescription}

This section first presents the problem description, which gives a high-level
view of the problem to be solved.  This is followed by the solution characteristics
specification, which presents the assumptions, theories, definitions and finally
the instance models.

\subsection{Problem Description} \label{Sec_pd}

BSS is intended to simulate the motion of a binary star system under
mutual gravitational interaction. Given the physical properties of two stars
and their initial conditions, the software determines how the positions of the
stars evolve over time according to a simplified physical model.

\subsubsection{Terminology and  Definitions}

This subsection provides a list of terms that are used in the subsequent
sections and their meaning, with the purpose of reducing ambiguity and making it
easier to correctly understand the requirements:

\begin{itemize}

\item binary star system: a system consisting of two stars that orbit around
their common center of mass due to gravitational interaction.

\item star: a massive astronomical object that is treated as a point mass in
the context of this document.

\item gravitational interaction: the mutual attractive force between two masses
as described by Newtonian gravity.

\item initial conditions: the positions and velocities of the stars at the
start of the simulation.

\item trajectory: the path traced by a star in space as a function of time.

\item center of mass: the point representing the average position of the mass
distribution of the system.

\item inertial reference frame: a reference frame in which Newton’s laws of
motion are valid without the introduction of fictitious forces.

\item simulation time span: the duration over which the evolution of the system
is computed.

\end{itemize}


\subsubsection{Physical System Description} \label{sec_phySystDescrip}

An overview of the physical system considered in this document is shown in
Figure~\ref{Fig_PhysicalSystem}. The system consists of two stars interacting
through mutual gravitational attraction. The stars are modeled as point masses
moving in a two-dimensional inertial reference frame.

The physical system is described at an initial time by the positions and
velocities of the two stars. Motion is considered relative to a fixed inertial
frame, and distances are measured using a Cartesian coordinate system. The
center of mass of the system follows inertial motion, while each star orbits
around the center of mass according to the gravitational interaction.
The orbit shown in the figure is illustrative only. The actual trajectory
depends on the initial conditions and is not necessarily elliptical.

The physical system examined in this document therefore consists of the
following components:
\begin{itemize}
\item[PS1:] star 1, modeled as a point mass,
\item[PS2:] star 2, modeled as a point mass,
\item[PS3:] the Newtonian gravitational interaction between the two stars, and
\item[PS4:] an inertial, two-dimensional Cartesian reference frame in which the motion is observed.
\end{itemize}

No external forces are applied to the system. As a result, the total linear
momentum of the system is conserved, and the motion is fully determined by the
initial conditions.

\begin{figure}[h!]
\begin{center}
 \includegraphics[width=0.6\textwidth]{PhysicalSystemDecription}
\caption{Simplified Binary Star System Model}
\label{Fig_PhysicalSystem}
\end{center}
\end{figure}

\subsubsection{Goal Statements} \label{sec_GoalStatements}

\noindent Given the masses, initial positions, and initial velocities of two stars, the goal statements are:

\begin{itemize}

\item[GS\refstepcounter{goalnum}\thegoalnum \label{G_positionsOverTime}:]
  Determine the positions of both stars as functions of time over a
  specified simulation interval.

\end{itemize}

\subsection{Solution Characteristics Specification}

The instance models that govern BSS are presented in
Subsection~\ref{sec_instance}.  The information to understand the meaning of the
instance models and their derivation is also presented, so that the instance
models can be verified.

\subsubsection{Assumptions} \label{sec_assumpt}

This section simplifies the original problem and helps in developing the
theoretical model by filling in the missing information for the physical system.
The numbers given in the square brackets refer to the theoretical model [TM],
general definition [GD], data definition [DD], instance model [IM], or likely
change [LC], in which the respective assumption is used.

\begin{itemize}

\item[A\refstepcounter{assumpnum}\theassumpnum \label{A_twoBody}:]
  The system consists of exactly two stars; third-body gravitational perturbations
  are ignored.

\item[A\refstepcounter{assumpnum}\theassumpnum \label{A_isolated}:]
  Non-gravitational forces (e.g., drag, thrust, radiation pressure) are neglected;
  the only force modeled is mutual gravitation between the two stars.

\item[A\refstepcounter{assumpnum}\theassumpnum \label{A_newtonianGravity}:]
  The gravitational interaction between the stars is modeled using Newton's law
  of universal gravitation.

\item[A\refstepcounter{assumpnum}\theassumpnum \label{A_nonRelativistic}:]
  The motion is modeled using classical (non-relativistic) mechanics; relativistic
  effects are neglected.

\item[A\refstepcounter{assumpnum}\theassumpnum \label{A_pointMass}:]
  Each star is modeled as a point mass, and effects due to stellar size,
  deformation, or rotation are neglected.

\item[A\refstepcounter{assumpnum}\theassumpnum \label{A_constantMass}:]
  The masses of the stars remain constant over time.

\item[A\refstepcounter{assumpnum}\theassumpnum \label{A_inertialFrame}:]
  The simulation is performed in an inertial reference frame.

\item[A\refstepcounter{assumpnum}\theassumpnum \label{A_planar}:]
  The motion of the binary star system is confined to a two-dimensional plane.

\item[A\refstepcounter{assumpnum}\theassumpnum \label{A_nonzeroSeparation}:]
  Collisions are out of scope: the separation distance satisfies
  $r_{12}(t) > 0$ for all simulated times (\dref{GD:RelPosSep}),
  so the gravitational force model remains well-defined.

\end{itemize}

\subsubsection{Theoretical Models}\label{sec_theoretical}

This section focuses on the general equations and laws that BSS is based
on.
~\newline

\noindent
\deftheory
% #2 refname of theory
{TM1}
% #3 label
{Center-of-Mass Reference Frame}
% #4 equation
{
  $\mathbf{R} = \dfrac{m_1 \mathbf{r}_1 + m_2 \mathbf{r}_2}{m_1 + m_2}$
}
% #5 description
{
  The above equation defines the center-of-mass (COM) position $\mathbf{R}$ of a
  two-body system, where $m_1$ and $m_2$ are the masses of the two stars and
  $\mathbf{r}_1$ and $\mathbf{r}_2$ are their position vectors.
  In the COM reference frame, the origin is chosen such that $\mathbf{R}=\mathbf{0}$.
}
% #6 Notes
{
None.
}
% #7 Source
{
  \citep{Taylor2005}
}
% #8 Referenced by
{
  \iref{IM:Orbit}
}
% #9 Preconditions
{
None.
}
% #1 derivation - not applicable by default
{}
\noindent
\deftheory
{TM2}
{Kinematics: Position, Velocity, and Acceleration}
{
  $\mathbf{v}(t) = \dfrac{d\mathbf{r}(t)}{dt}$,
  \quad
  $\mathbf{a}(t) = \dfrac{d\mathbf{v}(t)}{dt} = \dfrac{d^2\mathbf{r}(t)}{dt^2}$
}
{
  This model defines velocity $\mathbf{v}(t)$ and acceleration $\mathbf{a}(t)$ as
  time derivatives of the position vector $\mathbf{r}(t)$.
}
{
None.
}
{
  \citep{Taylor2005}
}
{
  \iref{IM:Orbit}
}
{
None.
}
{}
\noindent
\deftheory
{TM3}
{Newton's Law of Universal Gravitation}
{
  $\mathbf{F}_{12} = - G \frac{m_1 m_2}{r_{12}^3}\,\mathbf{r}_{12}$,
  where $\mathbf{r}_{12}=\mathbf{r}_{12}(t)$ and $r_{12}=r_{12}(t)$
  (\dref{GD:RelPosSep}).
}
{
  The above equation gives the gravitational force exerted between two stars of
  masses $m_1$ and $m_2$. The force is proportional to the product of the masses
  and inversely proportional to the square of the distance between them, acting
  along the line joining their positions.
}
{
None.
}
{
  \citep{Taylor2005}
}
{
  \tref{TM4}
}
{
None.
}
{}
\noindent
\deftheory
{TM4}
{Equations of Motion for a Two-Body System}
{
  $m_1 \mathbf{a}_1 = \mathbf{F}_{12},
  \quad
  m_2 \mathbf{a}_2 = -\mathbf{F}_{12}$
}
{
  This model relates the motion of each star to the gravitational force acting
  on it according to Newton's second law of motion. Together with
  \tref{TM3}, these equations govern the dynamics of the binary star system.
}
{
None.
}
{
  \citep{Taylor2005}
}
{
  \iref{IM:Orbit}
}
{
None.
}
{}

\plt{``Ref.\ By'' is used repeatedly with the different types of information.
  This stands for Referenced By.  It means that the models, definitions and
  assumptions listed reference the current model, definition or assumption.
  This information is given for traceability.  Ref. By provides a pointer in the
  opposite direction to what we commonly do.  You still need to have a reference
  in the other direction pointing to the current model, definition or
  assumption.  As an example, if TM1 is referenced by GD2, that means that GD2 will
  explicitly include a reference to TM1.}

~\newline

\subsubsection{General Definitions}\label{sec_gendef}

This section collects the laws and equations that will be used in building the instance models.

~\newline

\noindent
\begin{minipage}{\textwidth}
\renewcommand*{\arraystretch}{1.5}
\begin{tabular}{| p{\colAwidth} | p{\colBwidth}|}
\hline
\rowcolor[gray]{0.9}
Number& GD\refstepcounter{defnum}\thedefnum \label{GD:RelPosSep}\\
\hline
Label &\bf Relative position and separation \\
\hline
SI Units&\si{\metre}\\
\hline
Equation&
$\mathbf{r}_{12}(t) = \mathbf{r}_1(t) - \mathbf{r}_2(t)$, \\
& $r_{12}(t) = \lVert \mathbf{r}_{12}(t) \rVert$
\\
\hline
Description &
$\mathbf{r}_{12}(t)$ is the relative position vector from star 2 to star 1, and
$r_{12}(t)$ is the corresponding separation distance.
\\
\hline
  Source & --- \\
  \hline
  Ref.\ By & \tref{TM3}, \iref{IM:Orbit}\\
  \hline
\end{tabular}
\end{minipage}\\

~\newline



\subsubsection{Data Definitions}\label{sec_datadef}

This section collects and defines all the data needed to build the instance
models. The dimension of each quantity is also given.\\

\noindent
\begin{minipage}{\textwidth}
\renewcommand*{\arraystretch}{1.5}
\begin{tabular}{| p{\colAwidth} | p{\colBwidth}|}
\hline
\rowcolor[gray]{0.9}
Number& DD\refstepcounter{datadefnum}\thedatadefnum \label{DD:G}\\
\hline
Label& \bf Gravitational constant\\
\hline
Symbol &$G$\\
\hline
SI Units & \si{\metre\cubed\per\kilogram\per\second\squared}\\
\hline
Equation&$G = 6.67430\times 10^{-11}\ \si{\metre\cubed\per\kilogram\per\second\squared}$\\
\hline
Description &
$G$ is the universal gravitational constant used in Newton's law of gravitation.
\\
\hline
Sources& \citep[p.~033105-41]{CODATA2018}\\
\hline
Ref.\ By & \tref{TM3}, \tref{TM4}, \iref{IM:Orbit}\\
\hline
\end{tabular}
\end{minipage}\\

\subsubsection{Data Types}\label{sec_datatypes}
The conventional data types used in undergraduate-level physics courses is sufficient. As such, the specifications define no extra data types.\\

\subsubsection{Instance Models} \label{sec_instance}    

\plt{The motivation for this section is to reduce the problem defined in
  ``Physical System Description'' (Section~\ref{sec_phySystDescrip}) to one
  expressed in mathematical terms. The IMs are built by refining the TMs and/or
  GDs.  This section should remain abstract.  The SRS should specify the
  requirements without considering the implementation.}

This section transforms the problem defined in Section~\ref{Sec_pd} into 
one which is expressed in mathematical terms. It uses concrete symbols defined 
in Section~\ref{sec_datadef} to replace the abstract symbols in the models 
identified in Sections~\ref{sec_theoretical} and~\ref{sec_gendef}. The goals \gsref{G_positionsOverTime} is solved by \iref{IM:Orbit}.

~\newline

%Instance Model 1

\noindent
\begin{minipage}{\textwidth}
\renewcommand*{\arraystretch}{1.5}
\begin{tabular}{| p{\colAwidth} | p{\colBwidth}|}
  \hline
  \rowcolor[gray]{0.9}
  Number & IM\refstepcounter{instnum}\theinstnum \label{IM:Orbit} \\
  \hline
  Label & \bf Binary Star Orbital Motion \\
  \hline
  Input &
  \begin{tabular}[t]{@{}l@{}}
    $m_1$, $m_2$, $\mathbf{r}_1(0)$, $\mathbf{r}_2(0)$,
    $\mathbf{v}_1(0)$, $\mathbf{v}_2(0)$, $t_\text{final}$
  \end{tabular} \\
  \hline
  Output &
  \begin{tabular}[t]{@{}l@{}}
    $\mathbf{r}_1(t)$, $\mathbf{r}_2(t)$ for $0 \le t \le t_\text{final}$,\\
    where the output is the solution of the following IVP:
  \end{tabular} \\
  &
  $m_1 \mathbf{a}_1(t) = - G \dfrac{m_1 m_2}{r_{12}(t)^3}\,\mathbf{r}_{12}(t),\quad
  m_2 \mathbf{a}_2(t) = \phantom{-} G \dfrac{m_1 m_2}{r_{12}(t)^3}\,\mathbf{r}_{12}(t)$, \\
  &
  \begin{tabular}[t]{@{}l@{}}
    with initial conditions $\mathbf{r}_1(0)$, $\mathbf{r}_2(0)$, $\mathbf{v}_1(0)$, $\mathbf{v}_2(0)$.
  \end{tabular} \\
  \hline
  Description &
  The above output is the solution of the above defined IVP. Here,\\
  &
  $\mathbf{a}_1(t)$ and $\mathbf{a}_2(t)$ are the acceleration vectors of stars 1 and 2
  (\si{\metre\per\second\squared}). \\
  &
  $m_1$ and $m_2$ are the masses of the two stars (\si{\kilogram}). \\
  &
  $\mathbf{r}_{12}(t)$ is the relative position vector from star 2 to star 1
  (\si{\metre}), and $r_{12}(t)=\lVert\mathbf{r}_{12}(t)\rVert$ is the corresponding
  separation distance.
  \\
  &
  $G$ is the gravitational constant
  (\si{\metre^3\per\kilogram\per\second^2}). \\
  &
  $t_\text{final}$ is the final simulation time (\si{\second}). \\
  \hline
  Sources & --- \\
  \hline
  Ref.\ By & --- \\
  \hline
\end{tabular}
\end{minipage}

%~\newline

\subsubsection*{Derivation of Binary Star Orbital Motion}

This model is a direct instantiation of \tref{TM3} and \tref{TM4} as an IVP for a
two-body system (\aref{A_twoBody}) in a planar setting (\aref{A_planar}). It
assumes Newtonian gravitation (\aref{A_newtonianGravity}) and classical mechanics
(\aref{A_nonRelativistic}), with point-mass stars (\aref{A_pointMass}) of constant
masses (\aref{A_constantMass}), in an inertial frame (\aref{A_inertialFrame}), and
neglects non-gravitational forces (\aref{A_isolated}). The relative position and
separation are defined by \dref{GD:RelPosSep}.

\subsubsection{Input Data Constraints} \label{sec_DataConstraints}    

Table~\ref{TblInputVar} lists the constraints on the input variables for BSS.
Physical constraints capture non-negotiable requirements (for example, positive
mass). Software constraints restrict inputs to ranges that are expected to be
numerically reasonable for the intended use of this simulator (Section~\ref{sec_Scope}).

The specification parameters referenced in the software constraints are listed in
Table~\ref{TblSpecParams}.

\begin{table}[!h]
  \caption{Input Variables} \label{TblInputVar}
  \renewcommand{\arraystretch}{1.2}
\noindent \begin{longtable*}{l l l l c}
  \toprule
  \textbf{Var} & \textbf{Physical Constraints} & \textbf{Software Constraints} &
                             \textbf{Typical Value} & \textbf{Uncertainty}\\
  \midrule
  $m_1$ & $m_1 > 0$ & $m_{\min} \le m_1 \le m_{\max}$ & $2.0\times 10^{30}\,\si{kg}$ & 5\% \\
  $m_2$ & $m_2 > 0$ & $m_{\min} \le m_2 \le m_{\max}$ & $1.6\times 10^{30}\,\si{kg}$ & 5\% \\
  $\mathbf{r}_1(0)$ & $\mathbf{r}_1(0)\in\mathbb{R}^2$ & $\lVert \mathbf{r}_1(0) \rVert \le r_{\max}$ & $(+7.5\times 10^{10},0)\,\si{m}$ & 1\% \\
  $\mathbf{r}_2(0)$ & $\mathbf{r}_2(0)\in\mathbb{R}^2$ & $\lVert \mathbf{r}_2(0) \rVert \le r_{\max}$ & $(-7.5\times 10^{10},0)\,\si{m}$ & 1\% \\
  $\mathbf{v}_1(0)$ & $\mathbf{v}_1(0)\in\mathbb{R}^2$ & $\lVert \mathbf{v}_1(0) \rVert \le v_{\max}$ & $(0,+1.0\times 10^{4})\,\si{m\per s}$ & 5\% \\
  $\mathbf{v}_2(0)$ & $\mathbf{v}_2(0)\in\mathbb{R}^2$ & $\lVert \mathbf{v}_2(0) \rVert \le v_{\max}$ & $(0,-1.0\times 10^{4})\,\si{m\per s}$ & 5\% \\
  $t_{\text{final}}$ & $t_{\text{final}} > 0$ & $t_{\min} \le t_{\text{final}} \le t_{\max}$ & $3.15\times 10^{7}\,\si{s}$ & --- \\
  \bottomrule
\end{longtable*}
\end{table}

\noindent
\begin{description}
\item[(*)] In the center-of-mass frame used by \iref{IM:Orbit}, the initial
positions must satisfy $m_1\mathbf{r}_1(0)+m_2\mathbf{r}_2(0)=\mathbf{0}$.
\end{description}

\begin{table}[!h]
\caption{Specification Parameter Values} \label{TblSpecParams}
\renewcommand{\arraystretch}{1.2}
\noindent \begin{longtable*}{l l}
  \toprule
  \textbf{Var} & \textbf{Value} \\
  \midrule
  $m_{\min}$ & $1.0\times 10^{29}\,\si{kg}$\\
  $m_{\max}$ & $1.0\times 10^{32}\,\si{kg}$\\
  $r_{\max}$ & $10^{13}\,\si{m}$\\
  $v_{\max}$ & $10^6\,\si{m\per s}$\\
  $t_{\min}$ & $10^3\,\si{s}$\\
  $t_{\max}$ & $10^{10}\,\si{s}$\\
  \bottomrule
\end{longtable*}
\end{table}

\subsubsection{Properties of a Correct Solution} \label{sec_CorrectSolution}

\noindent
A correct solution must exhibit with the limitations as defined in Table~\ref{TblOutputVar}.

\begin{table}[!h]
\caption{Output Variables} \label{TblOutputVar}
\renewcommand{\arraystretch}{1.2}
\noindent \begin{longtable*}{l l}
  \toprule
  \textbf{Var} & \textbf{Physical Constraints} \\
  \midrule
  $\mathbf{r}_1(t)$ & $\mathbf{r}_1(t) \in \mathbb{R}^2$ for $0 \le t \le t_{\text{final}}$ \\
  $\mathbf{r}_2(t)$ & $\mathbf{r}_2(t) \in \mathbb{R}^2$ for $0 \le t \le t_{\text{final}}$ \\
  \bottomrule
\end{longtable*}
\end{table}

\section{Requirements}

\plt{The requirements refine the goal statement.  They will make heavy use of
  references to the instance models.}

This section provides the functional requirements, the business tasks that the
software is expected to complete, and the nonfunctional requirements, the
qualities that the software is expected to exhibit.

\subsection{Functional Requirements}

\noindent \begin{itemize}

\item[R\refstepcounter{reqnum}\thereqnum \label{R_Inputs}:] The inputs shall be provided by the end user according to the data definitions
specified in Table~\ref{TblInputVar}.

\item[R\refstepcounter{reqnum}\thereqnum \label{R_OutputInputs}:] The inputs of BSS shall be displayed to the user before any meaningful
computation begins.

\item[R\refstepcounter{reqnum}\thereqnum \label{R_Calculate}:] The provided inputs shall be validated against the constraints specified in
Table~\ref{TblInputVar}.

\item[R\refstepcounter{reqnum}\thereqnum \label{R_VerifyOutput}:]
The software shall verify that the computed results satisfy the consistency
conditions defined by the instance models in Section~\ref{sec_instance}.

\item[R\refstepcounter{reqnum}\thereqnum \label{R_Output}:]
The software shall compute and output the system state variables specified in
Table~\ref{TblOutputVar} in accordance with the instance models described in Section~\ref{sec_instance}.


\end{itemize}

\plt{Every IM should map to at least one requirement, but not every requirement
  has to map to a corresponding IM.}

\subsection{Nonfunctional Requirements}

\plt{List your nonfunctional requirements.  You may consider using a fit
  criterion to make them verifiable.}
\plt{The goal is for the nonfunctional requirements to be unambiguous, abstract
  and verifiable.  This isn't easy to show succinctly, so a good strategy may be
to give a ``high level'' view of the requirement, but allow for the details to
be covered in the Verification and Validation document.}
\plt{An absolute requirement on a quality of the system is rarely needed.  For
  instance, an accuracy of 0.0101 \% is likely fine, even if the requirement is
  for 0.01 \% accuracy.  Therefore, the emphasis will often be more on
  describing now well the quality is achieved, through experimentation, and
  possibly theory, rather than meeting some bar that was defined a priori.}
\plt{You do not need an entry for correctness in your NFRs.  The purpose of the
  SRS is to record the requirements that need to be satisfied for correctness.
  Any statement of correctness would just be redundant. Rather than discuss
  correctness, you can characterize how far away from the correct (true)
  solution you are allowed to be.  This is discussed under accuracy.}

\noindent \begin{itemize}

\item[NFR\refstepcounter{nfrnum}\thenfrnum \label{NFR_Accuracy}:]
  \textbf{Accuracy} The accuracy of the computed solutions shall meet the level required for the
intended scientific and educational usage domain. Confidence in the accuracy of
the software shall be established by following the procedures outlined in the
Verification and Validation Plan.

\item[NFR\refstepcounter{nfrnum}\thenfrnum \label{NFR_Usability}:] \textbf{Usability}
  The outputs of BSS should be easy to inspect and reuse. In particular,
the software should export results in a consistent format so that users can
post-process and visualize trajectories with external tools. The usability of
the software will be discussed and evaluated following the procedure in the
Verification and Validation Plan.

\item[NFR\refstepcounter{nfrnum}\thenfrnum \label{NFR_Maintainability}:]
  \textbf{Maintainability} The specification and models are encoded using Drasil. This should make routine
updates (e.g., adjusting parameters or refining assumptions) manageable and
predictable. When changes are made, their impact on the generated artifacts
should be clear and traceable.

\item[NFR\refstepcounter{nfrnum}\thenfrnum \label{NFR_Portability}:]
  \textbf{Portability} BSS should run on common operating systems, including Linux, macOS, and
Windows.

\end{itemize}

\subsection{Rationale}

The scope of this document is intentionally limited to a simplified binary star
system in order to keep the problem tractable and clearly defined. Restricting
the system to two bodies in a two-dimensional inertial reference frame allows
the essential physical behavior to be captured without introducing unnecessary
complexity.

The modeling choices and assumptions are based on standard practices in
introductory classical mechanics. Newtonian gravitation is used because it is
sufficient for describing the intended class of systems and is widely
understood by the intended readers. Additional effects such as relativistic
corrections and external forces are excluded to maintain focus on the core
problem.

Typical values and representations are chosen to support clarity, consistency,
and ease of verification. The use of Drasil motivates explicit and structured
definitions of assumptions and models, which improves traceability between the
physical problem, the mathematical formulation, and the resulting software
artifacts.


\section{Likely Changes}    

\noindent \begin{itemize}

\item[LC\refstepcounter{lcnum}\thelcnum\label{LC_meaningfulLabel}:] Additional derived output quantities, such as total energy
or angular momentum, may be added to support validation and analysis of the
simulation results.
\end{itemize}

\section{Unlikely Changes}    

There are no unlikely changes.

\section{Traceability Matrices and Graphs}

The purpose of the traceability matrices is to provide easy references on what
has to be additionally modified if a certain component is changed.  Every time a
component is changed, the items in the column of that component that are marked
with an ``X'' may have to be modified as well.  Table~\ref{Table:trace} shows the
dependencies of theoretical models, general definitions, data definitions, and
instance models with each other. Table~\ref{Table:R_trace} shows the
dependencies of instance models, requirements, and data constraints on each
other. Table~\ref{Table:A_trace} shows the dependencies of theoretical models,
general definitions, data definitions, instance models, and likely changes on
the assumptions.

\plt{You will have to modify these tables for your problem.}

\plt{The traceability matrix is not generally symmetric.  If GD1 uses A1, that
  means that GD1's derivation or presentation requires invocation of A1.  A1
  does not use GD1.  A1 is ``used by'' GD1.}

\plt{The traceability matrix is challenging to maintain manually.  Please do
  your best.  In the future tools (like Drasil) will make this much easier.}

\afterpage{
\begin{landscape}
\begin{table}[h!]
\centering
\begin{tabular}{|c|c|c|c|c|c|c|c|c|c|}
\hline
	& \aref{A_twoBody}& \aref{A_isolated}& \aref{A_newtonianGravity}& \aref{A_nonRelativistic}& \aref{A_pointMass}& \aref{A_constantMass}& \aref{A_inertialFrame}& \aref{A_planar}& \aref{A_nonzeroSeparation} \\
\hline
\tref{TM1}            & & & & & & & & & \\ \hline
\tref{TM2}            & & & & & & & & & \\ \hline
\tref{TM3}            & & & X& & & & & & X\\ \hline
\tref{TM4}            & & & & X& & & X& & \\ \hline
\dref{GD:RelPosSep}   & & & & & & & & X& \\ \hline
\ddref{DD:G}          & & & & & & & & & \\ \hline
\iref{IM:Orbit}       & X& X& X& X& X& X& X& X& X\\
\hline
\lcref{LC_meaningfulLabel} & X& X& X& X& X& X& X& X& X\\
\hline
\end{tabular}
\caption{Traceability Matrix Showing the Connections Between Assumptions and Other Items}
\label{Table:A_trace}
\end{table}
\end{landscape}
}

\begin{table}[h!]
\centering
\begin{tabular}{|c|c|c|c|c|c|c|c|}
\hline
	& \tref{TM1} & \tref{TM2} & \tref{TM3} & \tref{TM4} & \dref{GD:RelPosSep} & \ddref{DD:G} & \iref{IM:Orbit} \\
\hline
\tref{TM1}          &  &  &  &  &  &  &  \\ \hline
\tref{TM2}          &  &  &  &  &  &  &  \\ \hline
\tref{TM3}          &  &  &  &  & X& X&  \\ \hline
\tref{TM4}          &  &  & X&  &  &  &  \\ \hline
\dref{GD:RelPosSep} &  &  &  &  &  &  &  \\ \hline
\ddref{DD:G}        &  &  &  &  &  &  &  \\ \hline
\iref{IM:Orbit}     & X& X& X& X& X& X&  \\ \hline
\end{tabular}
\caption{Traceability Matrix Showing the Connections Between Items of Different Sections}
\label{Table:trace}
\end{table}

\begin{table}[h!]
\centering
\begin{tabular}{|c|c|c|c|c|c|c|c|}
\hline
	& \iref{IM:Orbit} & \ref{sec_DataConstraints} & \rref{R_Inputs} & \rref{R_OutputInputs} & \rref{R_Calculate} & \rref{R_VerifyOutput} & \rref{R_Output} \\
\hline
\iref{IM:Orbit}                &  & X& X&  & X&  & X\\ \hline
\rref{R_Inputs}                &  & X&  &  &  &  &  \\ \hline
\rref{R_OutputInputs}          & X&  & X&  &  &  & X\\ \hline
\rref{R_Calculate}             & X& X& X&  &  &  &  \\ \hline
\rref{R_VerifyOutput}          & X&  &  &  & X&  & X\\ \hline
\rref{R_Output}                & X&  &  &  &  &  &  \\ \hline
\end{tabular}
\caption{Traceability Matrix Showing the Connections Between Instance Models, Requirements, and Data Constraints}
\label{Table:R_trace}
\end{table}

% \begin{figure}[h!]
% 	\begin{center}
% 		%\rotatebox{-90}
% 		{
% 			\includegraphics[width=\textwidth]{ATrace.png}
% 		}
% 		\caption{\label{Fig_ATrace} Traceability Matrix Showing the Connections Between Items of Different Sections}
% 	\end{center}
% \end{figure}


% \begin{figure}[h!]
% 	\begin{center}
% 		%\rotatebox{-90}
% 		{
% 			\includegraphics[width=0.7\textwidth]{RTrace.png}
% 		}
% 		\caption{\label{Fig_RTrace} Traceability Matrix Showing the Connections Between Requirements, Instance Models, and Data Constraints}
% 	\end{center}
% \end{figure}

\section{Development Plan}

The software shall be developed using Drasil. Drasil will be used to encode the
problem knowledge (assumptions, models, and requirements) and to generate the
software artifacts in a traceable way.

Drasil will also be used to generate and maintain this SRS document. Developers
should follow the Drasil workflow and documentation, and make changes by updating
the encoded knowledge in Drasil rather than editing generated artifacts directly.


\section{Values of Auxiliary Constants}

\noindent \textbf{Gravitational constant:}
$G = 6.67430\times 10^{-11}\ \si{\metre^3\per\kilogram\per\second^2}$.

\plt{Show the values of the symbolic parameters introduced in the report.}

\plt{The definition of the requirements will likely call for SYMBOLIC\_CONSTANTS.
Their values are defined in this section for easy maintenance.}

\plt{The value of FRACTION, for the Maintainability NFR would be given here.}

\newpage

\bibliographystyle {plainnat}
\bibliography{../../refs/References}

\end{document}
